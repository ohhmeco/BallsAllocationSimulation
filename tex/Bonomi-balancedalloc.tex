\documentclass{article}
\usepackage[T1]{fontenc}
\usepackage[utf8]{inputenc}
\usepackage{amsmath}
\usepackage{amssymb}
\usepackage{graphicx}
\usepackage{float}
\usepackage[english]{babel}
\usepackage{booktabs}
\usepackage{siunitx}
\usepackage{geometry}

\geometry{
 a4paper,
 total={170mm,257mm},
 left=20mm,
 top=20mm,
}

\sisetup{
    round-mode = places,
    round-precision = 2,
    table-format = 2.2,
    table-align-text-after = false,
    group-separator = {}
}

\title{Analysis of Balls into Bins Allocation Strategies}
\author{Randomized Algorithms Assessment}
\date{\today}

\begin{document}
\begin{center}
  \textbf{Link directory: \url{https://github.com/ohhmeco/BallsAllocationSimulation}}
\end{center}
\maketitle

This document presents the results of a simulation comparing various randomized strategies for the Balls into Bins problem in a heavy-loaded scenario ($n=M^2$). The goal 
is to evaluate how limiting the decision-making process (reduced choice as $1$-Choice , partial knowledge like $(1+\beta)$-Choice or Binary Queries, or delayed information 
such as in the $b$-Batched case) affects the average maximum load, defined as the \textbf{Gap} ($\bar{G}_n$). The simulation was run with $M=100$ bins and $N_{MAX}=10,000$ balls, 
with $T=30$ repetitions for averaging the gap through different experiments.

\section{Experimental Setup}

The simulation parameters are defined as follows:
\begin{itemize}
    \item \textbf{Number of Bins ($M$)}: 100
    \item \textbf{Maximum Number of Balls ($N_{MAX}$)}: 10,000 ($M^2$)
    \item \textbf{Number of Runs ($T$)}: 30
\end{itemize}

The Gap ($\bar{G}_n$) is calculated as the maximum load minus the average load: $\bar{G}_n = \max(X_i) - n/M$. The final results are summarized at $n=N_{MAX}=10,000$ and in file named "output" present in the project's main directory.

\section{Results: Comparative Analysis of Strategies}

The strategies are grouped into three main categories. The reference strategies, $1$-Choice and $2$-Choice, define the "performance" bounds.

\subsection{Standard Strategies (Perfect Information)}
These experiments use real-time, perfect load information (equivalent to $b=1$). The results confirm the theoretical advantage of two choices (2-Choice) over one choice (1-Choice), which reduces the gap.

\begin{table}[H]
    \centering
    \caption{Gap Comparison for Standard and $(1+\beta)$-Choice Strategies at $n=10,000$}
    \label{tab:standard}
    \begin{tabular}{lc}
        \toprule
        Strategy & Average Gap $\bar{G}_{N_{MAX}}$ \\
        \midrule
        1-Choice & \num{26.13} \\
        2-Choice & \num{1.83} \\
        (1+0.2)-Choice & \num{12.07} \\
        (1+0.5)-Choice & \num{4.40} \\
        (1+0.8)-Choice & \num{2.43} \\
        \bottomrule
    \end{tabular}
\end{table}

The (1+0.8)-Choice strategy performs very close to the optimal 2-Choice (Gap $\approx 2.43$ 
vs $1.83$). Conversely, increasing the probability of a random choice, as seen in \textbf{(1
+0.2)-Choice}, causes a worst performance, approaching the $1$-Choice gap.

\subsection{Non updated Information (b-Batched)}

This section examines how delayed load information affects the 2-Choice strategy. The batch size $b$ is the interval after which the load vector is updated.

\begin{table}[H]
    \centering
    \caption{Gap Comparison for b-Batched Strategies at $n=10,000$}
    \label{tab:batched}
    \begin{tabular}{lc}
        \toprule
        Strategy & Average Gap $\bar{G}_{N_{MAX}}$ \\
        \midrule
        2-Choice (Standard, $b=1$) & \num{1.83} \\
        2-Choice ($b=100/M$) & \num{3.30} \\
        2-Choice ($b=1000/M$) & \num{11.77} \\
        \bottomrule
    \end{tabular}
\end{table}

A small batch size ($b=M$) nearly doubles the gap ($\approx 3.30$). A large batch size 
($b=10M$) leads to a clear performance degradation (Gap $\approx 11.77$), demonstrating 
that the benefit of 2-Choice is sensitive to the freshness of the load information.

\subsection{Impact of Partial Information (Binary Query)}

This section evaluates the 2-Choice strategy using only $k$ binary queries to infer load 
information (median, quartiles).

\begin{table}[H]
    \centering
    \caption{Gap Comparison for Binary Query Strategies at $n=10,000$}
    \label{tab:query}
    \begin{tabular}{lc}
        \toprule
        Strategy & Average Gap $\bar{G}_{N_{MAX}}$ \\
        \midrule
        2-Choice (Standard, $k=\infty$) & \num{1.83} \\
        2-Choice ($k=1$ Query) & \num{6.60} \\
        2-Choice ($k=2$ Query) & \num{3.73} \\
        \bottomrule
    \end{tabular}
\end{table}

The $k=2$ Query strategy, which uses quartiles to better discriminate candidates, provides 
a good trade-off (Gap $\approx 3.73$). While obviously being worse than perfect knowledge, 
it is better than the $k=1$ Query strategy (Gap $\approx 6.60$), showing that even a little 
more information can improve the performance.

\newpage

\section{Graphical Results}

The following figures illustrate the evolution of the average gap ($\bar{G}_n$) for the three
experimental settings, including the shaded areas, that represent the standard deviation (uncertainty) across $T=30$ runs.

\begin{figure}[H]
    \centering
    \includegraphics[width=\textwidth]{../Plots/grafico_standard.png}
    \caption{Gap Evolution in Standard Setting ($T=30$). Shows that 2-Choice is optimal and performance degrades as the probability of random choice (1-Choice) increases.}
    \label{fig:standard}
\end{figure}

\begin{figure}[H]
    \centering
    \includegraphics[width=\textwidth]{../Plots/grafico_batched.png}
    \caption{Gap Evolution in b-Batched Setting ($T=30$). Shows that by increasing the batch size $b$ and so by delaying the information update, we compromise the performance of the 2-Choice strategy.}
    \label{fig:batched}
\end{figure}

\begin{figure}[H]
    \centering
    \includegraphics[width=\textwidth]{../Plots/grafico_query.png}
    \caption{Gap Evolution in Binary Query Setting ($T=30$). Shows that 2 Binary Queries provide better results than 1 Query ($k=1$) as they approach the 2-Choice behavior.}
    \label{fig:query}
\end{figure}

\end{document}
